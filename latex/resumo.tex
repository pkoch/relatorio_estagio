%  Last modified on  Wed May 24 11:00:09 2006
\thispagestyle{plain}
\section*{Resumo}

Imagine-se um dispositivo \nname{Bluetooth} a ser localizado, apelidado de
\foreign{Rover}, com nenhum ou muito pouco software espec�fico, dentro de uma �rea
aonde est� instalada uma grelha de v�rios outros dispositivos de \nname{Bluetooth},
apelidados de \foreign{Beacons}. Os \foreign{Beacons} s�o estacion�rios,
� conhecida a sua localiza��o, e est�o ligados a um computador central. O
\foreign{Rover} � transportado por uma pessoa e, por isso, assume-se que tem
liberdade de movimentos total.

O objectivo deste est�gio era construir uma plataforma que dote a grelha de
\foreign{Beacons} com a capacidade de localizar o \foreign{Rover} quando este se
encontra dentro da grelha.

\textcolor[rgb]{1,0,0}{RESUMIR O RESTO DO CONTE�DO DO DOCUMENTO.}

%$\ll$M�ximo 1 p�gina ou 350 palavras.
%\vspace*{\baselineskip}
%
%``Um resumo � uma representa��o abreviada e precisa de um documento,
%sem acrescento de interpreta��o ou cr�tica, escrita de forma
%impessoal'' (ISO 214).
%
%O resumo pode ter, por exemplo, as seguintes 3 componentes:
%\begin{itemize}
%\item 1 par�grafo inicial de introdu��o do contexto geral do trabalho.
%\item Resumo dos aspectos mais importantes do trabalho descrito no presente %%relat�rio, que por sua vez documenta o trabalho mais importante realizado %%durante o est�gio. Deve mencionar tudo aquilo que foi feito, por isso deve %%concentrar-se no que � realmente importante e que deve ajudar o leitor a %%decidir se deve ou n�o consultar o restante relat�rio.
%\item 1 par�grafo final com as conclus�es do trabalho realizado.
%\end{itemize}
%$\gg$