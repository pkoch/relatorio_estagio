%  Last modified on  Wed May 24 11:03:47 2006
\thispagestyle{plain}

\section*{Agradecimentos}
Agrade�o a todas as pessoas da Novabase que toleraram coabitar comigo no meu local
de trabalho, distraindo-me e ajudando-me sempre que precisei. Obrigado Daniel
Carneiro, Ricardo Afonso, Hugo Barrote, M�rio Couto e Jo�o Cavadas.

Agrade�o aos meus orientadores por todo o trabalho que envergaram e por todas as
vezes que discutiram comigo, trazendo-me sempre � terra. Obrigado Jo�o Correia
Lopes e Jo�o Vasco Ranito.

Agrade�o � FEUP por todo o meu percurso acad�mico que culminou com a oportunidade
de fazer parte de uma equipa da vida real, com problemas de vida real, dando-me
experi�ncia da vida real. Obrigado, comunidade FEUP e institui��o FEUP.

Agrade�o tamb�m ao PRODEP por todo o esfor�o desenvolvido para criar melhores condi��es
de est�gio para os estudantes do ensino superior portugu�s. Em nome de todos os estudantes,
muito obrigado PRODEP.

%$\ll$M�ximo 1 p�gina.
%\vspace*{\baselineskip}
%
%Agradecimentos a todas as pessoas da institui��o de est�gio que
%estiveram directamente envolvidas no trabalho realizado, ou que
%contribu�ram para o seu sucesso.
%
%Agradecimentos a todas as pessoas da FEUP que de algum modo apoiaram a
%realiza��o do trabalho, ou que contribu�ram para o seu sucesso.
%
%Eventuais agradecimentos a outras pessoas, incluindo amigos ou
%familiares.
%
%Finalmente, referir e agradecer o financiamento do PRODEP (saber junto
%do secretariado da LEIC qual � o n�mero do projecto PRODEP), caso
%aplic�vel.
%
%\vspace*{\baselineskip}
%\noindent$\gg$
