%  Last modified on  Wed May 24 17:51:57 2006

\chapter{Conclus�es e Perspectivas de Trabalho Futuro} \label{ch:conclusao}

$\ll$3 ou 5 p�ginas sugeridas
\vspace*{\baselineskip}

De prefer�ncia, n�o exceder as 80 p�ginas no total do Relat�rio de
Est�gio at� ao fim desta �ltima sec��o.

Algumas alus�es devem ser feitas aos cap�tulos anteriores e,
principalmente, �s principais conclus�es de cada um.

As conclus�es finais devem focar o sucesso/insucesso do trabalho,
revendo as dificuldades encontradas. 
Devem resumir, de alguma forma, as vantagens do produto desenvolvido e
a utilidade que possa ter para a institui��o de est�gio ou para os
seus clientes/parceiros.

Algumas quest�es podem ainda ser inclu�das acerca da forma como o
est�gio correu; a integra��o, a forma��o dada pela institui��o, as
facilidades e dificuldades sentidas ao longo do est�gio.

\vspace*{\baselineskip}
\noindent$\gg$
